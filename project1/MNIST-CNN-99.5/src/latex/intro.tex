\cleardoublepage
\setcounter{chapter}{0} %The counter for chapters should be set one less than 
                        %the actual chapter number, for example, {1} for 
                        %chapter 2, and {2} for chapter 3.
\setcounter{section}{1} %The counter for sections should be set to match the
                        %actual chapter number, for example, {2} for sections 
                        %in chapter, and {3} for sections in chapter 3. 

\chapter{Introduction}
\label{intro}
\markboth{WEBGUI USERS' GUIDE}{INTRODUCTION} %For running heads. The first 
         %field will be the same for every chapter. The second will contain the 
         %appropriate chapter title.
\pagenumbering{arabic}
\setcounter{page}{1} 

\section{Description}

\f{WEBGUI} is a C language library that creates for your software a graphical user interface (GUI) after
only a few library function calls. \f{WEBGUI} is platform independent because it displays everything
in any standard web browser. And \f{WEBGUI} allows the interface to reside on either the local machine
or a remote machine.
 

\section{Overview}

\f{WEBGUI} works by enabling your program to mimic a web server. When your program calls the library
routine "webstart(int x)", a new thread is created which sets up a web server and begins listening on port x.  
Both your program and the web server run on the same host machine. Using any standard web browser either 
locally or remotely, enter the URL of the host machine. Then, the web browser will display a set of controls. 
The web browser is in constant communication with
your software running on the host machine. All of your software's output will be displayed in the web
browser and all input you provide in the web browser will be sent to your software.

\section{Installation}

Simply compile and link \f{WEBGUI} to your program. No other files are needed. For example, 
"gcc YourProgram.c webgui.c". To start the service, call the routine "webstart(int port)" within your program. 
Next, to send text output to the web browser, call the routine "writeline(char* str)" where str is a string of length 80. 
To receive input, call the routine "readline(char* str)" where str is a buffer that can receive a string of length 80.
Lastly, to display 2D images, or 3D objects, call either "webimagedisplay" or "webgldisplay".
If you would like your web browser controls to have buttons, call the routine "webinit". To learn more
about these and other routines, read Chapter \ref{chap:2} titled, "WebGUI API" below.
